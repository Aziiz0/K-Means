\documentclass[11pt]{article}

    \usepackage[breakable]{tcolorbox}
    \usepackage{parskip} % Stop auto-indenting (to mimic markdown behaviour)
    

    % Basic figure setup, for now with no caption control since it's done
    % automatically by Pandoc (which extracts ![](path) syntax from Markdown).
    \usepackage{graphicx}
    % Maintain compatibility with old templates. Remove in nbconvert 6.0
    \let\Oldincludegraphics\includegraphics
    % Ensure that by default, figures have no caption (until we provide a
    % proper Figure object with a Caption API and a way to capture that
    % in the conversion process - todo).
    \usepackage{caption}
    \DeclareCaptionFormat{nocaption}{}
    \captionsetup{format=nocaption,aboveskip=0pt,belowskip=0pt}

    \usepackage{float}
    \floatplacement{figure}{H} % forces figures to be placed at the correct location
    \usepackage{xcolor} % Allow colors to be defined
    \usepackage{enumerate} % Needed for markdown enumerations to work
    \usepackage{geometry} % Used to adjust the document margins
    \usepackage{amsmath} % Equations
    \usepackage{amssymb} % Equations
    \usepackage{textcomp} % defines textquotesingle
    % Hack from http://tex.stackexchange.com/a/47451/13684:
    \AtBeginDocument{%
        \def\PYZsq{\textquotesingle}% Upright quotes in Pygmentized code
    }
    \usepackage{upquote} % Upright quotes for verbatim code
    \usepackage{eurosym} % defines \euro

    \usepackage{iftex}
    \ifPDFTeX
        \usepackage[T1]{fontenc}
        \IfFileExists{alphabeta.sty}{
              \usepackage{alphabeta}
          }{
              \usepackage[mathletters]{ucs}
              \usepackage[utf8x]{inputenc}
          }
    \else
        \usepackage{fontspec}
        \usepackage{unicode-math}
    \fi

    \usepackage{fancyvrb} % verbatim replacement that allows latex
    \usepackage{grffile} % extends the file name processing of package graphics
                         % to support a larger range
    \makeatletter % fix for old versions of grffile with XeLaTeX
    \@ifpackagelater{grffile}{2019/11/01}
    {
      % Do nothing on new versions
    }
    {
      \def\Gread@@xetex#1{%
        \IfFileExists{"\Gin@base".bb}%
        {\Gread@eps{\Gin@base.bb}}%
        {\Gread@@xetex@aux#1}%
      }
    }
    \makeatother
    \usepackage[Export]{adjustbox} % Used to constrain images to a maximum size
    \adjustboxset{max size={0.9\linewidth}{0.9\paperheight}}

    % The hyperref package gives us a pdf with properly built
    % internal navigation ('pdf bookmarks' for the table of contents,
    % internal cross-reference links, web links for URLs, etc.)
    \usepackage{hyperref}
    % The default LaTeX title has an obnoxious amount of whitespace. By default,
    % titling removes some of it. It also provides customization options.
    \usepackage{titling}
    \usepackage{longtable} % longtable support required by pandoc >1.10
    \usepackage{booktabs}  % table support for pandoc > 1.12.2
    \usepackage{array}     % table support for pandoc >= 2.11.3
    \usepackage{calc}      % table minipage width calculation for pandoc >= 2.11.1
    \usepackage[inline]{enumitem} % IRkernel/repr support (it uses the enumerate* environment)
    \usepackage[normalem]{ulem} % ulem is needed to support strikethroughs (\sout)
                                % normalem makes italics be italics, not underlines
    \usepackage{mathrsfs}
    

    
    % Colors for the hyperref package
    \definecolor{urlcolor}{rgb}{0,.145,.698}
    \definecolor{linkcolor}{rgb}{.71,0.21,0.01}
    \definecolor{citecolor}{rgb}{.12,.54,.11}

    % ANSI colors
    \definecolor{ansi-black}{HTML}{3E424D}
    \definecolor{ansi-black-intense}{HTML}{282C36}
    \definecolor{ansi-red}{HTML}{E75C58}
    \definecolor{ansi-red-intense}{HTML}{B22B31}
    \definecolor{ansi-green}{HTML}{00A250}
    \definecolor{ansi-green-intense}{HTML}{007427}
    \definecolor{ansi-yellow}{HTML}{DDB62B}
    \definecolor{ansi-yellow-intense}{HTML}{B27D12}
    \definecolor{ansi-blue}{HTML}{208FFB}
    \definecolor{ansi-blue-intense}{HTML}{0065CA}
    \definecolor{ansi-magenta}{HTML}{D160C4}
    \definecolor{ansi-magenta-intense}{HTML}{A03196}
    \definecolor{ansi-cyan}{HTML}{60C6C8}
    \definecolor{ansi-cyan-intense}{HTML}{258F8F}
    \definecolor{ansi-white}{HTML}{C5C1B4}
    \definecolor{ansi-white-intense}{HTML}{A1A6B2}
    \definecolor{ansi-default-inverse-fg}{HTML}{FFFFFF}
    \definecolor{ansi-default-inverse-bg}{HTML}{000000}

    % common color for the border for error outputs.
    \definecolor{outerrorbackground}{HTML}{FFDFDF}

    % commands and environments needed by pandoc snippets
    % extracted from the output of `pandoc -s`
    \providecommand{\tightlist}{%
      \setlength{\itemsep}{0pt}\setlength{\parskip}{0pt}}
    \DefineVerbatimEnvironment{Highlighting}{Verbatim}{commandchars=\\\{\}}
    % Add ',fontsize=\small' for more characters per line
    \newenvironment{Shaded}{}{}
    \newcommand{\KeywordTok}[1]{\textcolor[rgb]{0.00,0.44,0.13}{\textbf{{#1}}}}
    \newcommand{\DataTypeTok}[1]{\textcolor[rgb]{0.56,0.13,0.00}{{#1}}}
    \newcommand{\DecValTok}[1]{\textcolor[rgb]{0.25,0.63,0.44}{{#1}}}
    \newcommand{\BaseNTok}[1]{\textcolor[rgb]{0.25,0.63,0.44}{{#1}}}
    \newcommand{\FloatTok}[1]{\textcolor[rgb]{0.25,0.63,0.44}{{#1}}}
    \newcommand{\CharTok}[1]{\textcolor[rgb]{0.25,0.44,0.63}{{#1}}}
    \newcommand{\StringTok}[1]{\textcolor[rgb]{0.25,0.44,0.63}{{#1}}}
    \newcommand{\CommentTok}[1]{\textcolor[rgb]{0.38,0.63,0.69}{\textit{{#1}}}}
    \newcommand{\OtherTok}[1]{\textcolor[rgb]{0.00,0.44,0.13}{{#1}}}
    \newcommand{\AlertTok}[1]{\textcolor[rgb]{1.00,0.00,0.00}{\textbf{{#1}}}}
    \newcommand{\FunctionTok}[1]{\textcolor[rgb]{0.02,0.16,0.49}{{#1}}}
    \newcommand{\RegionMarkerTok}[1]{{#1}}
    \newcommand{\ErrorTok}[1]{\textcolor[rgb]{1.00,0.00,0.00}{\textbf{{#1}}}}
    \newcommand{\NormalTok}[1]{{#1}}

    % Additional commands for more recent versions of Pandoc
    \newcommand{\ConstantTok}[1]{\textcolor[rgb]{0.53,0.00,0.00}{{#1}}}
    \newcommand{\SpecialCharTok}[1]{\textcolor[rgb]{0.25,0.44,0.63}{{#1}}}
    \newcommand{\VerbatimStringTok}[1]{\textcolor[rgb]{0.25,0.44,0.63}{{#1}}}
    \newcommand{\SpecialStringTok}[1]{\textcolor[rgb]{0.73,0.40,0.53}{{#1}}}
    \newcommand{\ImportTok}[1]{{#1}}
    \newcommand{\DocumentationTok}[1]{\textcolor[rgb]{0.73,0.13,0.13}{\textit{{#1}}}}
    \newcommand{\AnnotationTok}[1]{\textcolor[rgb]{0.38,0.63,0.69}{\textbf{\textit{{#1}}}}}
    \newcommand{\CommentVarTok}[1]{\textcolor[rgb]{0.38,0.63,0.69}{\textbf{\textit{{#1}}}}}
    \newcommand{\VariableTok}[1]{\textcolor[rgb]{0.10,0.09,0.49}{{#1}}}
    \newcommand{\ControlFlowTok}[1]{\textcolor[rgb]{0.00,0.44,0.13}{\textbf{{#1}}}}
    \newcommand{\OperatorTok}[1]{\textcolor[rgb]{0.40,0.40,0.40}{{#1}}}
    \newcommand{\BuiltInTok}[1]{{#1}}
    \newcommand{\ExtensionTok}[1]{{#1}}
    \newcommand{\PreprocessorTok}[1]{\textcolor[rgb]{0.74,0.48,0.00}{{#1}}}
    \newcommand{\AttributeTok}[1]{\textcolor[rgb]{0.49,0.56,0.16}{{#1}}}
    \newcommand{\InformationTok}[1]{\textcolor[rgb]{0.38,0.63,0.69}{\textbf{\textit{{#1}}}}}
    \newcommand{\WarningTok}[1]{\textcolor[rgb]{0.38,0.63,0.69}{\textbf{\textit{{#1}}}}}


    % Define a nice break command that doesn't care if a line doesn't already
    % exist.
    \def\br{\hspace*{\fill} \\* }
    % Math Jax compatibility definitions
    \def\gt{>}
    \def\lt{<}
    \let\Oldtex\TeX
    \let\Oldlatex\LaTeX
    \renewcommand{\TeX}{\textrm{\Oldtex}}
    \renewcommand{\LaTeX}{\textrm{\Oldlatex}}
    % Document parameters
    % Document title
    \title{Final Project}
    
    
    
    
    
% Pygments definitions
\makeatletter
\def\PY@reset{\let\PY@it=\relax \let\PY@bf=\relax%
    \let\PY@ul=\relax \let\PY@tc=\relax%
    \let\PY@bc=\relax \let\PY@ff=\relax}
\def\PY@tok#1{\csname PY@tok@#1\endcsname}
\def\PY@toks#1+{\ifx\relax#1\empty\else%
    \PY@tok{#1}\expandafter\PY@toks\fi}
\def\PY@do#1{\PY@bc{\PY@tc{\PY@ul{%
    \PY@it{\PY@bf{\PY@ff{#1}}}}}}}
\def\PY#1#2{\PY@reset\PY@toks#1+\relax+\PY@do{#2}}

\@namedef{PY@tok@w}{\def\PY@tc##1{\textcolor[rgb]{0.73,0.73,0.73}{##1}}}
\@namedef{PY@tok@c}{\let\PY@it=\textit\def\PY@tc##1{\textcolor[rgb]{0.24,0.48,0.48}{##1}}}
\@namedef{PY@tok@cp}{\def\PY@tc##1{\textcolor[rgb]{0.61,0.40,0.00}{##1}}}
\@namedef{PY@tok@k}{\let\PY@bf=\textbf\def\PY@tc##1{\textcolor[rgb]{0.00,0.50,0.00}{##1}}}
\@namedef{PY@tok@kp}{\def\PY@tc##1{\textcolor[rgb]{0.00,0.50,0.00}{##1}}}
\@namedef{PY@tok@kt}{\def\PY@tc##1{\textcolor[rgb]{0.69,0.00,0.25}{##1}}}
\@namedef{PY@tok@o}{\def\PY@tc##1{\textcolor[rgb]{0.40,0.40,0.40}{##1}}}
\@namedef{PY@tok@ow}{\let\PY@bf=\textbf\def\PY@tc##1{\textcolor[rgb]{0.67,0.13,1.00}{##1}}}
\@namedef{PY@tok@nb}{\def\PY@tc##1{\textcolor[rgb]{0.00,0.50,0.00}{##1}}}
\@namedef{PY@tok@nf}{\def\PY@tc##1{\textcolor[rgb]{0.00,0.00,1.00}{##1}}}
\@namedef{PY@tok@nc}{\let\PY@bf=\textbf\def\PY@tc##1{\textcolor[rgb]{0.00,0.00,1.00}{##1}}}
\@namedef{PY@tok@nn}{\let\PY@bf=\textbf\def\PY@tc##1{\textcolor[rgb]{0.00,0.00,1.00}{##1}}}
\@namedef{PY@tok@ne}{\let\PY@bf=\textbf\def\PY@tc##1{\textcolor[rgb]{0.80,0.25,0.22}{##1}}}
\@namedef{PY@tok@nv}{\def\PY@tc##1{\textcolor[rgb]{0.10,0.09,0.49}{##1}}}
\@namedef{PY@tok@no}{\def\PY@tc##1{\textcolor[rgb]{0.53,0.00,0.00}{##1}}}
\@namedef{PY@tok@nl}{\def\PY@tc##1{\textcolor[rgb]{0.46,0.46,0.00}{##1}}}
\@namedef{PY@tok@ni}{\let\PY@bf=\textbf\def\PY@tc##1{\textcolor[rgb]{0.44,0.44,0.44}{##1}}}
\@namedef{PY@tok@na}{\def\PY@tc##1{\textcolor[rgb]{0.41,0.47,0.13}{##1}}}
\@namedef{PY@tok@nt}{\let\PY@bf=\textbf\def\PY@tc##1{\textcolor[rgb]{0.00,0.50,0.00}{##1}}}
\@namedef{PY@tok@nd}{\def\PY@tc##1{\textcolor[rgb]{0.67,0.13,1.00}{##1}}}
\@namedef{PY@tok@s}{\def\PY@tc##1{\textcolor[rgb]{0.73,0.13,0.13}{##1}}}
\@namedef{PY@tok@sd}{\let\PY@it=\textit\def\PY@tc##1{\textcolor[rgb]{0.73,0.13,0.13}{##1}}}
\@namedef{PY@tok@si}{\let\PY@bf=\textbf\def\PY@tc##1{\textcolor[rgb]{0.64,0.35,0.47}{##1}}}
\@namedef{PY@tok@se}{\let\PY@bf=\textbf\def\PY@tc##1{\textcolor[rgb]{0.67,0.36,0.12}{##1}}}
\@namedef{PY@tok@sr}{\def\PY@tc##1{\textcolor[rgb]{0.64,0.35,0.47}{##1}}}
\@namedef{PY@tok@ss}{\def\PY@tc##1{\textcolor[rgb]{0.10,0.09,0.49}{##1}}}
\@namedef{PY@tok@sx}{\def\PY@tc##1{\textcolor[rgb]{0.00,0.50,0.00}{##1}}}
\@namedef{PY@tok@m}{\def\PY@tc##1{\textcolor[rgb]{0.40,0.40,0.40}{##1}}}
\@namedef{PY@tok@gh}{\let\PY@bf=\textbf\def\PY@tc##1{\textcolor[rgb]{0.00,0.00,0.50}{##1}}}
\@namedef{PY@tok@gu}{\let\PY@bf=\textbf\def\PY@tc##1{\textcolor[rgb]{0.50,0.00,0.50}{##1}}}
\@namedef{PY@tok@gd}{\def\PY@tc##1{\textcolor[rgb]{0.63,0.00,0.00}{##1}}}
\@namedef{PY@tok@gi}{\def\PY@tc##1{\textcolor[rgb]{0.00,0.52,0.00}{##1}}}
\@namedef{PY@tok@gr}{\def\PY@tc##1{\textcolor[rgb]{0.89,0.00,0.00}{##1}}}
\@namedef{PY@tok@ge}{\let\PY@it=\textit}
\@namedef{PY@tok@gs}{\let\PY@bf=\textbf}
\@namedef{PY@tok@gp}{\let\PY@bf=\textbf\def\PY@tc##1{\textcolor[rgb]{0.00,0.00,0.50}{##1}}}
\@namedef{PY@tok@go}{\def\PY@tc##1{\textcolor[rgb]{0.44,0.44,0.44}{##1}}}
\@namedef{PY@tok@gt}{\def\PY@tc##1{\textcolor[rgb]{0.00,0.27,0.87}{##1}}}
\@namedef{PY@tok@err}{\def\PY@bc##1{{\setlength{\fboxsep}{\string -\fboxrule}\fcolorbox[rgb]{1.00,0.00,0.00}{1,1,1}{\strut ##1}}}}
\@namedef{PY@tok@kc}{\let\PY@bf=\textbf\def\PY@tc##1{\textcolor[rgb]{0.00,0.50,0.00}{##1}}}
\@namedef{PY@tok@kd}{\let\PY@bf=\textbf\def\PY@tc##1{\textcolor[rgb]{0.00,0.50,0.00}{##1}}}
\@namedef{PY@tok@kn}{\let\PY@bf=\textbf\def\PY@tc##1{\textcolor[rgb]{0.00,0.50,0.00}{##1}}}
\@namedef{PY@tok@kr}{\let\PY@bf=\textbf\def\PY@tc##1{\textcolor[rgb]{0.00,0.50,0.00}{##1}}}
\@namedef{PY@tok@bp}{\def\PY@tc##1{\textcolor[rgb]{0.00,0.50,0.00}{##1}}}
\@namedef{PY@tok@fm}{\def\PY@tc##1{\textcolor[rgb]{0.00,0.00,1.00}{##1}}}
\@namedef{PY@tok@vc}{\def\PY@tc##1{\textcolor[rgb]{0.10,0.09,0.49}{##1}}}
\@namedef{PY@tok@vg}{\def\PY@tc##1{\textcolor[rgb]{0.10,0.09,0.49}{##1}}}
\@namedef{PY@tok@vi}{\def\PY@tc##1{\textcolor[rgb]{0.10,0.09,0.49}{##1}}}
\@namedef{PY@tok@vm}{\def\PY@tc##1{\textcolor[rgb]{0.10,0.09,0.49}{##1}}}
\@namedef{PY@tok@sa}{\def\PY@tc##1{\textcolor[rgb]{0.73,0.13,0.13}{##1}}}
\@namedef{PY@tok@sb}{\def\PY@tc##1{\textcolor[rgb]{0.73,0.13,0.13}{##1}}}
\@namedef{PY@tok@sc}{\def\PY@tc##1{\textcolor[rgb]{0.73,0.13,0.13}{##1}}}
\@namedef{PY@tok@dl}{\def\PY@tc##1{\textcolor[rgb]{0.73,0.13,0.13}{##1}}}
\@namedef{PY@tok@s2}{\def\PY@tc##1{\textcolor[rgb]{0.73,0.13,0.13}{##1}}}
\@namedef{PY@tok@sh}{\def\PY@tc##1{\textcolor[rgb]{0.73,0.13,0.13}{##1}}}
\@namedef{PY@tok@s1}{\def\PY@tc##1{\textcolor[rgb]{0.73,0.13,0.13}{##1}}}
\@namedef{PY@tok@mb}{\def\PY@tc##1{\textcolor[rgb]{0.40,0.40,0.40}{##1}}}
\@namedef{PY@tok@mf}{\def\PY@tc##1{\textcolor[rgb]{0.40,0.40,0.40}{##1}}}
\@namedef{PY@tok@mh}{\def\PY@tc##1{\textcolor[rgb]{0.40,0.40,0.40}{##1}}}
\@namedef{PY@tok@mi}{\def\PY@tc##1{\textcolor[rgb]{0.40,0.40,0.40}{##1}}}
\@namedef{PY@tok@il}{\def\PY@tc##1{\textcolor[rgb]{0.40,0.40,0.40}{##1}}}
\@namedef{PY@tok@mo}{\def\PY@tc##1{\textcolor[rgb]{0.40,0.40,0.40}{##1}}}
\@namedef{PY@tok@ch}{\let\PY@it=\textit\def\PY@tc##1{\textcolor[rgb]{0.24,0.48,0.48}{##1}}}
\@namedef{PY@tok@cm}{\let\PY@it=\textit\def\PY@tc##1{\textcolor[rgb]{0.24,0.48,0.48}{##1}}}
\@namedef{PY@tok@cpf}{\let\PY@it=\textit\def\PY@tc##1{\textcolor[rgb]{0.24,0.48,0.48}{##1}}}
\@namedef{PY@tok@c1}{\let\PY@it=\textit\def\PY@tc##1{\textcolor[rgb]{0.24,0.48,0.48}{##1}}}
\@namedef{PY@tok@cs}{\let\PY@it=\textit\def\PY@tc##1{\textcolor[rgb]{0.24,0.48,0.48}{##1}}}

\def\PYZbs{\char`\\}
\def\PYZus{\char`\_}
\def\PYZob{\char`\{}
\def\PYZcb{\char`\}}
\def\PYZca{\char`\^}
\def\PYZam{\char`\&}
\def\PYZlt{\char`\<}
\def\PYZgt{\char`\>}
\def\PYZsh{\char`\#}
\def\PYZpc{\char`\%}
\def\PYZdl{\char`\$}
\def\PYZhy{\char`\-}
\def\PYZsq{\char`\'}
\def\PYZdq{\char`\"}
\def\PYZti{\char`\~}
% for compatibility with earlier versions
\def\PYZat{@}
\def\PYZlb{[}
\def\PYZrb{]}
\makeatother


    % For linebreaks inside Verbatim environment from package fancyvrb.
    \makeatletter
        \newbox\Wrappedcontinuationbox
        \newbox\Wrappedvisiblespacebox
        \newcommand*\Wrappedvisiblespace {\textcolor{red}{\textvisiblespace}}
        \newcommand*\Wrappedcontinuationsymbol {\textcolor{red}{\llap{\tiny$\m@th\hookrightarrow$}}}
        \newcommand*\Wrappedcontinuationindent {3ex }
        \newcommand*\Wrappedafterbreak {\kern\Wrappedcontinuationindent\copy\Wrappedcontinuationbox}
        % Take advantage of the already applied Pygments mark-up to insert
        % potential linebreaks for TeX processing.
        %        {, <, #, %, $, ' and ": go to next line.
        %        _, }, ^, &, >, - and ~: stay at end of broken line.
        % Use of \textquotesingle for straight quote.
        \newcommand*\Wrappedbreaksatspecials {%
            \def\PYGZus{\discretionary{\char`\_}{\Wrappedafterbreak}{\char`\_}}%
            \def\PYGZob{\discretionary{}{\Wrappedafterbreak\char`\{}{\char`\{}}%
            \def\PYGZcb{\discretionary{\char`\}}{\Wrappedafterbreak}{\char`\}}}%
            \def\PYGZca{\discretionary{\char`\^}{\Wrappedafterbreak}{\char`\^}}%
            \def\PYGZam{\discretionary{\char`\&}{\Wrappedafterbreak}{\char`\&}}%
            \def\PYGZlt{\discretionary{}{\Wrappedafterbreak\char`\<}{\char`\<}}%
            \def\PYGZgt{\discretionary{\char`\>}{\Wrappedafterbreak}{\char`\>}}%
            \def\PYGZsh{\discretionary{}{\Wrappedafterbreak\char`\#}{\char`\#}}%
            \def\PYGZpc{\discretionary{}{\Wrappedafterbreak\char`\%}{\char`\%}}%
            \def\PYGZdl{\discretionary{}{\Wrappedafterbreak\char`\$}{\char`\$}}%
            \def\PYGZhy{\discretionary{\char`\-}{\Wrappedafterbreak}{\char`\-}}%
            \def\PYGZsq{\discretionary{}{\Wrappedafterbreak\textquotesingle}{\textquotesingle}}%
            \def\PYGZdq{\discretionary{}{\Wrappedafterbreak\char`\"}{\char`\"}}%
            \def\PYGZti{\discretionary{\char`\~}{\Wrappedafterbreak}{\char`\~}}%
        }
        % Some characters . , ; ? ! / are not pygmentized.
        % This macro makes them "active" and they will insert potential linebreaks
        \newcommand*\Wrappedbreaksatpunct {%
            \lccode`\~`\.\lowercase{\def~}{\discretionary{\hbox{\char`\.}}{\Wrappedafterbreak}{\hbox{\char`\.}}}%
            \lccode`\~`\,\lowercase{\def~}{\discretionary{\hbox{\char`\,}}{\Wrappedafterbreak}{\hbox{\char`\,}}}%
            \lccode`\~`\;\lowercase{\def~}{\discretionary{\hbox{\char`\;}}{\Wrappedafterbreak}{\hbox{\char`\;}}}%
            \lccode`\~`\:\lowercase{\def~}{\discretionary{\hbox{\char`\:}}{\Wrappedafterbreak}{\hbox{\char`\:}}}%
            \lccode`\~`\?\lowercase{\def~}{\discretionary{\hbox{\char`\?}}{\Wrappedafterbreak}{\hbox{\char`\?}}}%
            \lccode`\~`\!\lowercase{\def~}{\discretionary{\hbox{\char`\!}}{\Wrappedafterbreak}{\hbox{\char`\!}}}%
            \lccode`\~`\/\lowercase{\def~}{\discretionary{\hbox{\char`\/}}{\Wrappedafterbreak}{\hbox{\char`\/}}}%
            \catcode`\.\active
            \catcode`\,\active
            \catcode`\;\active
            \catcode`\:\active
            \catcode`\?\active
            \catcode`\!\active
            \catcode`\/\active
            \lccode`\~`\~
        }
    \makeatother

    \let\OriginalVerbatim=\Verbatim
    \makeatletter
    \renewcommand{\Verbatim}[1][1]{%
        %\parskip\z@skip
        \sbox\Wrappedcontinuationbox {\Wrappedcontinuationsymbol}%
        \sbox\Wrappedvisiblespacebox {\FV@SetupFont\Wrappedvisiblespace}%
        \def\FancyVerbFormatLine ##1{\hsize\linewidth
            \vtop{\raggedright\hyphenpenalty\z@\exhyphenpenalty\z@
                \doublehyphendemerits\z@\finalhyphendemerits\z@
                \strut ##1\strut}%
        }%
        % If the linebreak is at a space, the latter will be displayed as visible
        % space at end of first line, and a continuation symbol starts next line.
        % Stretch/shrink are however usually zero for typewriter font.
        \def\FV@Space {%
            \nobreak\hskip\z@ plus\fontdimen3\font minus\fontdimen4\font
            \discretionary{\copy\Wrappedvisiblespacebox}{\Wrappedafterbreak}
            {\kern\fontdimen2\font}%
        }%

        % Allow breaks at special characters using \PYG... macros.
        \Wrappedbreaksatspecials
        % Breaks at punctuation characters . , ; ? ! and / need catcode=\active
        \OriginalVerbatim[#1,codes*=\Wrappedbreaksatpunct]%
    }
    \makeatother

    % Exact colors from NB
    \definecolor{incolor}{HTML}{303F9F}
    \definecolor{outcolor}{HTML}{D84315}
    \definecolor{cellborder}{HTML}{CFCFCF}
    \definecolor{cellbackground}{HTML}{F7F7F7}

    % prompt
    \makeatletter
    \newcommand{\boxspacing}{\kern\kvtcb@left@rule\kern\kvtcb@boxsep}
    \makeatother
    \newcommand{\prompt}[4]{
        {\ttfamily\llap{{\color{#2}[#3]:\hspace{3pt}#4}}\vspace{-\baselineskip}}
    }
    

    
    % Prevent overflowing lines due to hard-to-break entities
    \sloppy
    % Setup hyperref package
    \hypersetup{
      breaklinks=true,  % so long urls are correctly broken across lines
      colorlinks=true,
      urlcolor=urlcolor,
      linkcolor=linkcolor,
      citecolor=citecolor,
      }
    % Slightly bigger margins than the latex defaults
    
    \geometry{verbose,tmargin=1in,bmargin=1in,lmargin=1in,rmargin=1in}
    
    

\begin{document}
    
    \maketitle
    
    

    
    {Introduction}

    In this lecture, we will explore the K-means clustering algorithm, a
popular unsupervised machine learning technique for partitioning data
into groups or clusters based on similarity. The K-means algorithm has
numerous applications, including image compression, customer
segmentation, and anomaly detection.

    The lecture is organized as follows:

    {1. Mathematical Theory:}

We will discuss the mathematical foundation of the K-means algorithm,
including the objective function and the iterative process of updating
cluster centroids.

    {2. Computational Study using Julia:}

We will implement the K-means clustering algorithm in Julia using the
Clustering.jl package and provide examples of code snippets that run and
produce output.

    {3. Applications:}

We will explore various real-world applications of the K-means
algorithm, such as image compression and customer segmentation, and
demonstrate an example of image compression using Julia.

    {4. Participation Check:}

We will include a short, interactive exercise to help you explore the
impact of the number of clusters (K) on the K-means clustering
algorithm.

    {5. Mini-Homework:}

Finally, we will provide a mini-homework consisting of two questions
that require you to apply the K-means algorithm to different datasets
and answer questions related to the clustering results.

    By the end of this lecture, you will have a deep understanding of the
K-means clustering algorithm, its mathematical basis, and its practical
applications. You will also gain hands-on experience in implementing the
algorithm using Julia and analyzing the results.

    {Mathematical Theory}

    The K-means clustering algorithm aims to partition a dataset into K
distinct clusters by minimizing the sum of squared distances between
data points and their corresponding cluster centroids. Let's discuss the
mathematical foundation of the K-means algorithm in more detail.

    {1. Objective Function:}

The objective function for K-means clustering, also known as the
inertia, is defined as the sum of squared distances between data points
and their corresponding cluster centroids:

    \[
J(C_1, C_2, \dots, C_K) = \sum_{k=1}^{K} \sum_{x_i \in C_k} \| x_i - \mu_k \|^2
\]

    Here, \(C_k\) represents the \(k\)-th cluster, \(\mu_k\) is the centroid
of the \(k\)-th cluster, and \(x_i\) is a data point belonging to the
\(k\)-th cluster.

    {2. Iterative Process:}

The K-means algorithm is an iterative process that consists of two main
steps:

    \begin{enumerate}
\def\labelenumi{\alph{enumi}.}
\tightlist
\item
  {Assignment step:}
\end{enumerate}

Assign each data point to the nearest centroid, forming K clusters:

    \[
C_k = \{ x_i : \| x_i - \mu_k \| \leq \| x_i - \mu_j \|, \forall j, 1 \leq j \leq K \}
\]

    \begin{enumerate}
\def\labelenumi{\alph{enumi}.}
\setcounter{enumi}{1}
\tightlist
\item
  {Update step:}
\end{enumerate}

Update the centroids by computing the mean of all data points assigned
to the corresponding cluster:

    \[
\mu_k = \frac{1}{|C_k|} \sum_{x_i \in C_k} x_i
\]

    The algorithm repeats these steps until the centroids' positions
stabilize, or a predefined stopping criterion is met.

    In the next section, we will implement the K-means clustering algorithm
in Julia and provide examples of code snippets that run and produce
output, illustrating the mathematical concepts discussed in this
section.

    {Computational Study using Julia}

    In this section, we will implement the K-means clustering algorithm in
Julia using the Clustering.jl package. We will provide examples of code
snippets that run and produce output, illustrating the mathematical
concepts discussed in the previous section.

    {1. Installing and Loading the Package:}

First, install the Clustering.jl package and load it into your Jupyter
Notebook:

    \begin{tcolorbox}[breakable, size=fbox, boxrule=1pt, pad at break*=1mm,colback=cellbackground, colframe=cellborder]
\prompt{In}{incolor}{1}{\boxspacing}
\begin{Verbatim}[commandchars=\\\{\}]
\PY{c}{\PYZsh{} Install the package (only needed once)}
\PY{k}{using}\PY{+w}{ }\PY{n}{Pkg}
\PY{n}{Pkg}\PY{o}{.}\PY{n}{add}\PY{p}{(}\PY{l+s}{\PYZdq{}}\PY{l+s}{Clustering}\PY{l+s}{\PYZdq{}}\PY{p}{)}
\PY{n}{Pkg}\PY{o}{.}\PY{n}{add}\PY{p}{(}\PY{l+s}{\PYZdq{}}\PY{l+s}{GR}\PY{l+s}{\PYZdq{}}\PY{p}{)}
\PY{n}{Pkg}\PY{o}{.}\PY{n}{add}\PY{p}{(}\PY{l+s}{\PYZdq{}}\PY{l+s}{Plots}\PY{l+s}{\PYZdq{}}\PY{p}{)}
\PY{n}{Pkg}\PY{o}{.}\PY{n}{add}\PY{p}{(}\PY{l+s}{\PYZdq{}}\PY{l+s}{Images}\PY{l+s}{\PYZdq{}}\PY{p}{)}
\PY{n}{Pkg}\PY{o}{.}\PY{n}{add}\PY{p}{(}\PY{l+s}{\PYZdq{}}\PY{l+s}{ImageIO}\PY{l+s}{\PYZdq{}}\PY{p}{)}
\PY{n}{Pkg}\PY{o}{.}\PY{n}{add}\PY{p}{(}\PY{l+s}{\PYZdq{}}\PY{l+s}{QuartzImageIO}\PY{l+s}{\PYZdq{}}\PY{p}{)}
\PY{n}{Pkg}\PY{o}{.}\PY{n}{add}\PY{p}{(}\PY{l+s}{\PYZdq{}}\PY{l+s}{FileIO}\PY{l+s}{\PYZdq{}}\PY{p}{)}
\end{Verbatim}
\end{tcolorbox}

    \begin{Verbatim}[commandchars=\\\{\}]
\textcolor{ansi-green-intense}{\textbf{   Resolving}} package versions{\ldots}
\textcolor{ansi-green-intense}{\textbf{  No Changes}} to `\textasciitilde{}/.julia/environment/v1.8/Project.toml`
\textcolor{ansi-green-intense}{\textbf{  No Changes}} to `\textasciitilde{}/.julia/environment/v1.8/Manifest.toml`
\textcolor{ansi-green-intense}{\textbf{   Resolving}} package versions{\ldots}
\textcolor{ansi-green-intense}{\textbf{  No Changes}} to `\textasciitilde{}/.julia/environment/v1.8/Project.toml`
\textcolor{ansi-green-intense}{\textbf{  No Changes}} to `\textasciitilde{}/.julia/environment/v1.8/Manifest.toml`
\textcolor{ansi-green-intense}{\textbf{   Resolving}} package versions{\ldots}
\textcolor{ansi-green-intense}{\textbf{  No Changes}} to `\textasciitilde{}/.julia/environment/v1.8/Project.toml`
\textcolor{ansi-green-intense}{\textbf{  No Changes}} to `\textasciitilde{}/.julia/environment/v1.8/Manifest.toml`
\textcolor{ansi-green-intense}{\textbf{   Resolving}} package versions{\ldots}
\textcolor{ansi-green-intense}{\textbf{  No Changes}} to `\textasciitilde{}/.julia/environment/v1.8/Project.toml`
\textcolor{ansi-green-intense}{\textbf{  No Changes}} to `\textasciitilde{}/.julia/environment/v1.8/Manifest.toml`
\textcolor{ansi-green-intense}{\textbf{   Resolving}} package versions{\ldots}
\textcolor{ansi-green-intense}{\textbf{  No Changes}} to `\textasciitilde{}/.julia/environment/v1.8/Project.toml`
\textcolor{ansi-green-intense}{\textbf{  No Changes}} to `\textasciitilde{}/.julia/environment/v1.8/Manifest.toml`
\textcolor{ansi-green-intense}{\textbf{   Resolving}} package versions{\ldots}
\textcolor{ansi-green-intense}{\textbf{  No Changes}} to `\textasciitilde{}/.julia/environment/v1.8/Project.toml`
\textcolor{ansi-green-intense}{\textbf{  No Changes}} to `\textasciitilde{}/.julia/environment/v1.8/Manifest.toml`
\textcolor{ansi-green-intense}{\textbf{   Resolving}} package versions{\ldots}
\textcolor{ansi-green-intense}{\textbf{  No Changes}} to `\textasciitilde{}/.julia/environment/v1.8/Project.toml`
\textcolor{ansi-green-intense}{\textbf{  No Changes}} to `\textasciitilde{}/.julia/environment/v1.8/Manifest.toml`
    \end{Verbatim}

    \begin{tcolorbox}[breakable, size=fbox, boxrule=1pt, pad at break*=1mm,colback=cellbackground, colframe=cellborder]
\prompt{In}{incolor}{2}{\boxspacing}
\begin{Verbatim}[commandchars=\\\{\}]
\PY{c}{\PYZsh{} Load the package}
\PY{k}{using}\PY{+w}{ }\PY{n}{Clustering}\PY{p}{,}\PY{+w}{ }\PY{n}{GR}\PY{p}{,}\PY{+w}{ }\PY{n}{Plots}
\PY{k}{using}\PY{+w}{ }\PY{n}{Images}\PY{p}{,}\PY{+w}{ }\PY{n}{ImageIO}\PY{p}{,}\PY{+w}{ }\PY{n}{QuartzImageIO}\PY{p}{,}\PY{+w}{ }\PY{n}{FileIO}

\PY{k}{using}\PY{+w}{ }\PY{n}{Random}\PY{p}{,}\PY{+w}{ }\PY{n}{LinearAlgebra}
\end{Verbatim}
\end{tcolorbox}

    \begin{Verbatim}[commandchars=\\\{\}]
\textcolor{ansi-yellow-intense}{\textbf{┌ }}\textcolor{ansi-yellow-intense}{\textbf{Warning: }}QuartzImageIO.jl can only be
used on Apple macOS. Suggested usage is
\textcolor{ansi-yellow-intense}{\textbf{│ }}    @static if Sys.isapple()
\textcolor{ansi-yellow-intense}{\textbf{│ }}        using QuartzImageIO
\textcolor{ansi-yellow-intense}{\textbf{│ }}        \# QuartzImageIO specific code goes here
\textcolor{ansi-yellow-intense}{\textbf{│ }}    end
\textcolor{ansi-yellow-intense}{\textbf{└ }}\textcolor{ansi-black-intense}{@ QuartzImageIO
\textasciitilde{}/.julia/packages/QuartzImageIO/hnNSo/src/QuartzImageIO.jl:725}
    \end{Verbatim}

    {2. Creating a Sample Dataset:}

Generate a two-dimensional dataset with three distinct clusters:

    \begin{tcolorbox}[breakable, size=fbox, boxrule=1pt, pad at break*=1mm,colback=cellbackground, colframe=cellborder]
\prompt{In}{incolor}{3}{\boxspacing}
\begin{Verbatim}[commandchars=\\\{\}]
\PY{c}{\PYZsh{} Create a 2D dataset with 3 clusters}
\PY{n}{Random}\PY{o}{.}\PY{n}{seed!}\PY{p}{(}\PY{l+m+mi}{1234}\PY{p}{)}
\PY{n}{X}\PY{+w}{ }\PY{o}{=}\PY{+w}{ }\PY{n}{vcat}\PY{p}{(}\PY{n}{randn}\PY{p}{(}\PY{l+m+mi}{100}\PY{p}{,}\PY{+w}{ }\PY{l+m+mi}{2}\PY{p}{)}\PY{p}{,}\PY{+w}{ }\PY{n}{randn}\PY{p}{(}\PY{l+m+mi}{100}\PY{p}{,}\PY{+w}{ }\PY{l+m+mi}{2}\PY{p}{)}\PY{+w}{ }\PY{o}{.+}\PY{+w}{ }\PY{p}{[}\PY{l+m+mi}{4}\PY{+w}{ }\PY{l+m+mi}{0}\PY{p}{]}\PY{p}{,}\PY{+w}{ }\PY{n}{randn}\PY{p}{(}\PY{l+m+mi}{100}\PY{p}{,}\PY{+w}{ }\PY{l+m+mi}{2}\PY{p}{)}\PY{+w}{ }\PY{o}{.+}\PY{+w}{ }\PY{p}{[}\PY{l+m+mi}{0}\PY{+w}{ }\PY{l+m+mi}{4}\PY{p}{]}\PY{p}{)}
\end{Verbatim}
\end{tcolorbox}

            \begin{tcolorbox}[breakable, size=fbox, boxrule=.5pt, pad at break*=1mm, opacityfill=0]
\prompt{Out}{outcolor}{3}{\boxspacing}
\begin{Verbatim}[commandchars=\\\{\}]
300×2 Matrix\{Float64\}:
  0.970656    0.262456
 -0.979218   -0.022244
  0.901861   -0.391293
 -0.0328031   0.0276206
 -0.600792   -2.29076
 -1.44518    -0.668539
  2.70742    -0.784686
  1.52445     1.12899
  0.759804    0.211177
 -0.881437    0.714642
  0.705993    0.0340663
  1.09156     0.568671
  0.871498   -2.62623
  ⋮
  0.25323     4.79092
 -0.970117    4.52911
  1.38702     1.22907
  1.72698     3.17966
 -0.885273    1.81276
  0.828122    5.09404
  2.53705     3.89909
 -0.0283863   4.16711
 -0.717544    5.63942
 -0.862679    4.51742
  0.884708    3.8376
  1.8553      3.63704
\end{Verbatim}
\end{tcolorbox}
        
    {3. Running K-means Clustering:}

Apply the K-means clustering algorithm to the dataset with K = 3:

    \begin{tcolorbox}[breakable, size=fbox, boxrule=1pt, pad at break*=1mm,colback=cellbackground, colframe=cellborder]
\prompt{In}{incolor}{4}{\boxspacing}
\begin{Verbatim}[commandchars=\\\{\}]
\PY{c}{\PYZsh{} Perform K\PYZhy{}means clustering with K = 3}
\PY{n}{K}\PY{+w}{ }\PY{o}{=}\PY{+w}{ }\PY{l+m+mi}{3}
\PY{n}{result}\PY{+w}{ }\PY{o}{=}\PY{+w}{ }\PY{n}{kmeans}\PY{p}{(}\PY{n}{X}\PY{o}{\PYZsq{}}\PY{p}{,}\PY{+w}{ }\PY{n}{K}\PY{p}{)}
\end{Verbatim}
\end{tcolorbox}

            \begin{tcolorbox}[breakable, size=fbox, boxrule=.5pt, pad at break*=1mm, opacityfill=0]
\prompt{Out}{outcolor}{4}{\boxspacing}
\begin{Verbatim}[commandchars=\\\{\}]
KmeansResult\{Matrix\{Float64\}, Float64, Int64\}([-0.07147576699238503
3.939618202410333 -0.08541484900239217; -0.17631306235328226
-0.12789679575297358 4.093219559251867], [1, 1, 1, 1, 1, 1, 2, 1, 1, 1  …  1, 3,
1, 3, 3, 3, 3, 3, 3, 3], [1.2785571906851518, 0.8477339780716218,
0.9936005713681204, 0.043084511807942656, 4.7510750461609295,
2.1293417544410427, 1.9496752541015852, 4.250776383583823, 0.841174944878848,
1.4498385485413108  …  4.102315209196008, 4.119374634386386, 4.6186700273446,
1.836197853156186, 6.9150300669621885, 0.008711675095632643, 2.7903155299414877,
0.7840849804319063, 1.0064804744450093, 3.9744678993634714], [98, 105, 97], [98,
105, 97], 586.9992335244391, 4, true)
\end{Verbatim}
\end{tcolorbox}
        
    {4. Accessing Clustering Results:}

Extract the cluster assignments, centroids, and other relevant
information:

    \begin{tcolorbox}[breakable, size=fbox, boxrule=1pt, pad at break*=1mm,colback=cellbackground, colframe=cellborder]
\prompt{In}{incolor}{5}{\boxspacing}
\begin{Verbatim}[commandchars=\\\{\}]
\PY{c}{\PYZsh{} Get cluster assignments and centroids}
\PY{n}{cluster\PYZus{}assignments}\PY{+w}{ }\PY{o}{=}\PY{+w}{ }\PY{n}{Clustering}\PY{o}{.}\PY{n}{assignments}\PY{p}{(}\PY{n}{result}\PY{p}{)}
\PY{n}{centroids}\PY{+w}{ }\PY{o}{=}\PY{+w}{ }\PY{n}{transpose}\PY{p}{(}\PY{n}{result}\PY{o}{.}\PY{n}{centers}\PY{p}{)}
\end{Verbatim}
\end{tcolorbox}

            \begin{tcolorbox}[breakable, size=fbox, boxrule=.5pt, pad at break*=1mm, opacityfill=0]
\prompt{Out}{outcolor}{5}{\boxspacing}
\begin{Verbatim}[commandchars=\\\{\}]
3×2 transpose(::Matrix\{Float64\}) with eltype Float64:
 -0.0714758  -0.176313
  3.93962    -0.127897
 -0.0854148   4.09322
\end{Verbatim}
\end{tcolorbox}
        
    {5. Visualizing the Clustering Results:}

Plot the data points, cluster assignments, and centroids using the
Plots.jl package:

    \begin{tcolorbox}[breakable, size=fbox, boxrule=1pt, pad at break*=1mm,colback=cellbackground, colframe=cellborder]
\prompt{In}{incolor}{6}{\boxspacing}
\begin{Verbatim}[commandchars=\\\{\}]
\PY{c}{\PYZsh{} Plot the clustered data points and centroids}
\PY{n}{Plots}\PY{o}{.}\PY{n}{scatter}\PY{p}{(}\PY{n}{X}\PY{p}{[}\PY{o}{:}\PY{p}{,}\PY{+w}{ }\PY{l+m+mi}{1}\PY{p}{]}\PY{p}{,}\PY{+w}{ }\PY{n}{X}\PY{p}{[}\PY{o}{:}\PY{p}{,}\PY{+w}{ }\PY{l+m+mi}{2}\PY{p}{]}\PY{p}{,}\PY{+w}{ }\PY{n}{marker\PYZus{}z}\PY{o}{=}\PY{n}{result}\PY{o}{.}\PY{n}{assignments}\PY{p}{,}\PY{+w}{ }\PY{n}{color}\PY{o}{=}\PY{l+s+ss}{:auto}\PY{p}{,}\PY{+w}{ }\PY{n}{legend}\PY{o}{=}\PY{n+nb}{false}\PY{p}{)}
\PY{n}{Plots}\PY{o}{.}\PY{n}{scatter!}\PY{p}{(}\PY{n}{centroids}\PY{p}{[}\PY{o}{:}\PY{p}{,}\PY{+w}{ }\PY{l+m+mi}{1}\PY{p}{]}\PY{p}{,}\PY{+w}{ }\PY{n}{centroids}\PY{p}{[}\PY{o}{:}\PY{p}{,}\PY{+w}{ }\PY{l+m+mi}{2}\PY{p}{]}\PY{p}{,}\PY{+w}{ }\PY{n}{marker}\PY{o}{=}\PY{p}{(}\PY{l+s+ss}{:x}\PY{p}{,}\PY{+w}{ }\PY{l+m+mi}{10}\PY{p}{)}\PY{p}{,}\PY{+w}{ }\PY{n}{linewidth}\PY{o}{=}\PY{l+m+mi}{3}\PY{p}{,}\PY{+w}{ }\PY{n}{color}\PY{o}{=}\PY{l+s+ss}{:black}\PY{p}{,}\PY{+w}{ }\PY{n}{label}\PY{o}{=}\PY{l+s}{\PYZdq{}}\PY{l+s}{Centroids}\PY{l+s}{\PYZdq{}}\PY{p}{)}
\end{Verbatim}
\end{tcolorbox}
 
            
\prompt{Out}{outcolor}{6}{}
    
    \begin{center}
    \adjustimage{max size={0.9\linewidth}{0.9\paperheight}}{Final Project_files/Final Project_33_0.pdf}
    \end{center}
    { \hspace*{\fill} \\}
    

    In the next section, we will explore various real-world applications of
the K-means algorithm, such as image compression and customer
segmentation, and demonstrate an example of image compression using
Julia.

    {Applications}

    In this section, we will explore various real-world applications of the
K-means algorithm, such as image compression and customer segmentation,
and demonstrate an example of image compression using Julia.

    {1. Image Compression:}

K-means clustering can be used to reduce the number of colors in an
image, resulting in image compression. This is done by treating each
pixel's color as a data point and clustering these colors into a smaller
set of representative colors.

    {Image Compression Example using Julia:}

Let's demonstrate image compression using the K-means algorithm and
Julia:

    \begin{tcolorbox}[breakable, size=fbox, boxrule=1pt, pad at break*=1mm,colback=cellbackground, colframe=cellborder]
\prompt{In}{incolor}{7}{\boxspacing}
\begin{Verbatim}[commandchars=\\\{\}]
\PY{c}{\PYZsh{} Load an image}
\PY{n}{image}\PY{+w}{ }\PY{o}{=}\PY{+w}{ }\PY{n}{load}\PY{p}{(}\PY{l+s}{\PYZdq{}}\PY{l+s}{birds.jpg}\PY{l+s}{\PYZdq{}}\PY{p}{)}

\PY{c}{\PYZsh{} Convert the image to an array of color values}
\PY{n}{image\PYZus{}array}\PY{+w}{ }\PY{o}{=}\PY{+w}{ }\PY{n}{channelview}\PY{p}{(}\PY{n}{image}\PY{p}{)}

\PY{c}{\PYZsh{} Reshape the array to a 2D matrix with color values as rows}
\PY{n}{X}\PY{+w}{ }\PY{o}{=}\PY{+w}{ }\PY{n}{reshape}\PY{p}{(}\PY{n}{image\PYZus{}array}\PY{p}{,}\PY{+w}{ }\PY{p}{(}\PY{l+m+mi}{3}\PY{p}{,}\PY{+w}{ }\PY{o}{:}\PY{p}{)}\PY{p}{)}

\PY{c}{\PYZsh{} Perform K\PYZhy{}means clustering with K = 20}
\PY{n}{K}\PY{+w}{ }\PY{o}{=}\PY{+w}{ }\PY{l+m+mi}{3}
\PY{n}{result}\PY{+w}{ }\PY{o}{=}\PY{+w}{ }\PY{n}{kmeans}\PY{p}{(}\PY{n}{X}\PY{p}{,}\PY{+w}{ }\PY{n}{K}\PY{p}{)}

\PY{c}{\PYZsh{} Replace each pixel\PYZsq{}s color with the nearest centroid}
\PY{n}{compressed\PYZus{}colors}\PY{+w}{ }\PY{o}{=}\PY{+w}{ }\PY{n}{result}\PY{o}{.}\PY{n}{centers}\PY{p}{[}\PY{o}{:}\PY{p}{,}\PY{+w}{ }\PY{n}{Clustering}\PY{o}{.}\PY{n}{assignments}\PY{p}{(}\PY{n}{result}\PY{p}{)}\PY{p}{]}
\PY{n}{compressed\PYZus{}image\PYZus{}array}\PY{+w}{ }\PY{o}{=}\PY{+w}{ }\PY{n}{reshape}\PY{p}{(}\PY{n}{compressed\PYZus{}colors}\PY{p}{,}\PY{+w}{ }\PY{n}{size}\PY{p}{(}\PY{n}{image\PYZus{}array}\PY{p}{)}\PY{p}{)}

\PY{c}{\PYZsh{} Convert the array back to an image}
\PY{n}{compressed\PYZus{}image}\PY{+w}{ }\PY{o}{=}\PY{+w}{ }\PY{n}{colorview}\PY{p}{(}\PY{n}{RGB}\PY{p}{,}\PY{+w}{ }\PY{n}{compressed\PYZus{}image\PYZus{}array}\PY{p}{)}

\PY{c}{\PYZsh{} Save the compressed image}
\PY{n}{save}\PY{p}{(}\PY{l+s}{\PYZdq{}}\PY{l+s}{compressed\PYZus{}image.jpg}\PY{l+s}{\PYZdq{}}\PY{p}{,}\PY{+w}{ }\PY{n}{compressed\PYZus{}image}\PY{p}{)}
\end{Verbatim}
\end{tcolorbox}

    \begin{Verbatim}[commandchars=\\\{\}]
\textcolor{ansi-yellow-intense}{\textbf{┌ }}\textcolor{ansi-yellow-intense}{\textbf{Warning: }}The clustering cost increased
at iteration \#19
\textcolor{ansi-yellow-intense}{\textbf{└ }}\textcolor{ansi-black-intense}{@ Clustering
/ext/julia/depot/packages/Clustering/eBCMN/src/kmeans.jl:191}
    \end{Verbatim}

    

    {2. Customer Segmentation:}

K-means clustering can be used to segment customers based on their
behavior or preferences. By clustering customers into groups, businesses
can develop targeted marketing strategies, identify potential areas of
growth, and optimize resource allocation.

    For example, an e-commerce company may cluster customers based on their
browsing behavior, purchase history, and demographic information. Each
cluster may represent a distinct customer segment, such as
high-spending, bargain hunters, or infrequent shoppers.

    In the next section, we will include a short, interactive exercise to
help you explore the impact of the number of clusters (K) on the K-means
clustering algorithm.

    {Participation Check}

    In this short, interactive exercise, you will explore the impact of
varying the number of clusters (K) on the K-means clustering algorithm
using a synthetic 2D dataset.

    \textbf{Task:} Generate a synthetic 2D dataset with four distinct
clusters and apply the K-means algorithm using different values of K
(e.g., K = 2, 3, 4, 5). Plot the resulting cluster assignments and
centroids for each K value. Observe the clustering results and discuss
the importance of selecting an appropriate value for K.

    {Dataset Generation:}

Generate a synthetic 2D dataset with four distinct clusters:

    \begin{tcolorbox}[breakable, size=fbox, boxrule=1pt, pad at break*=1mm,colback=cellbackground, colframe=cellborder]
\prompt{In}{incolor}{8}{\boxspacing}
\begin{Verbatim}[commandchars=\\\{\}]
\PY{c}{\PYZsh{} Create a 2D dataset with 4 clusters}
\PY{n}{Random}\PY{o}{.}\PY{n}{seed!}\PY{p}{(}\PY{l+m+mi}{1234}\PY{p}{)}
\PY{n}{X}\PY{+w}{ }\PY{o}{=}\PY{+w}{ }\PY{n}{vcat}\PY{p}{(}\PY{n}{randn}\PY{p}{(}\PY{l+m+mi}{100}\PY{p}{,}\PY{+w}{ }\PY{l+m+mi}{2}\PY{p}{)}\PY{p}{,}\PY{+w}{ }\PY{n}{randn}\PY{p}{(}\PY{l+m+mi}{100}\PY{p}{,}\PY{+w}{ }\PY{l+m+mi}{2}\PY{p}{)}\PY{+w}{ }\PY{o}{.+}\PY{+w}{ }\PY{p}{[}\PY{l+m+mi}{4}\PY{+w}{ }\PY{l+m+mi}{0}\PY{p}{]}\PY{p}{,}\PY{+w}{ }\PY{n}{randn}\PY{p}{(}\PY{l+m+mi}{100}\PY{p}{,}\PY{+w}{ }\PY{l+m+mi}{2}\PY{p}{)}\PY{+w}{ }\PY{o}{.+}\PY{+w}{ }\PY{p}{[}\PY{l+m+mi}{0}\PY{+w}{ }\PY{l+m+mi}{4}\PY{p}{]}\PY{p}{,}\PY{+w}{ }\PY{n}{randn}\PY{p}{(}\PY{l+m+mi}{100}\PY{p}{,}\PY{+w}{ }\PY{l+m+mi}{2}\PY{p}{)}\PY{+w}{ }\PY{o}{.+}\PY{+w}{ }\PY{p}{[}\PY{l+m+mi}{4}\PY{+w}{ }\PY{l+m+mi}{4}\PY{p}{]}\PY{p}{)}

\PY{c}{\PYZsh{} Your solution here}
\end{Verbatim}
\end{tcolorbox}

            \begin{tcolorbox}[breakable, size=fbox, boxrule=.5pt, pad at break*=1mm, opacityfill=0]
\prompt{Out}{outcolor}{8}{\boxspacing}
\begin{Verbatim}[commandchars=\\\{\}]
400×2 Matrix\{Float64\}:
  0.970656    0.262456
 -0.979218   -0.022244
  0.901861   -0.391293
 -0.0328031   0.0276206
 -0.600792   -2.29076
 -1.44518    -0.668539
  2.70742    -0.784686
  1.52445     1.12899
  0.759804    0.211177
 -0.881437    0.714642
  0.705993    0.0340663
  1.09156     0.568671
  0.871498   -2.62623
  ⋮
  3.90369     5.28423
  4.00324     4.53129
  5.36942     4.35276
  4.10907     4.7954
  4.15615     2.10847
  3.01303     4.61578
  4.70232     4.59369
  4.32091     2.13339
  4.21332     4.97857
  4.00218     4.67994
  4.43465     4.38153
  3.81463     4.37
\end{Verbatim}
\end{tcolorbox}
        
    {K-means Clustering and Visualization:}

Apply the K-means algorithm for different K values and visualize the
results:

    \begin{tcolorbox}[breakable, size=fbox, boxrule=1pt, pad at break*=1mm,colback=cellbackground, colframe=cellborder]
\prompt{In}{incolor}{9}{\boxspacing}
\begin{Verbatim}[commandchars=\\\{\}]
\PY{c}{\PYZsh{} Create a 2D dataset with 4 clusters}
\PY{n}{Random}\PY{o}{.}\PY{n}{seed!}\PY{p}{(}\PY{l+m+mi}{1234}\PY{p}{)}
\PY{n}{X}\PY{+w}{ }\PY{o}{=}\PY{+w}{ }\PY{n}{vcat}\PY{p}{(}\PY{n}{randn}\PY{p}{(}\PY{l+m+mi}{100}\PY{p}{,}\PY{+w}{ }\PY{l+m+mi}{2}\PY{p}{)}\PY{p}{,}\PY{+w}{ }\PY{n}{randn}\PY{p}{(}\PY{l+m+mi}{100}\PY{p}{,}\PY{+w}{ }\PY{l+m+mi}{2}\PY{p}{)}\PY{+w}{ }\PY{o}{.+}\PY{+w}{ }\PY{p}{[}\PY{l+m+mi}{4}\PY{+w}{ }\PY{l+m+mi}{0}\PY{p}{]}\PY{p}{,}\PY{+w}{ }\PY{n}{randn}\PY{p}{(}\PY{l+m+mi}{100}\PY{p}{,}\PY{+w}{ }\PY{l+m+mi}{2}\PY{p}{)}\PY{+w}{ }\PY{o}{.+}\PY{+w}{ }\PY{p}{[}\PY{l+m+mi}{0}\PY{+w}{ }\PY{l+m+mi}{4}\PY{p}{]}\PY{p}{,}\PY{+w}{ }\PY{n}{randn}\PY{p}{(}\PY{l+m+mi}{100}\PY{p}{,}\PY{+w}{ }\PY{l+m+mi}{2}\PY{p}{)}\PY{+w}{ }\PY{o}{.+}\PY{+w}{ }\PY{p}{[}\PY{l+m+mi}{4}\PY{+w}{ }\PY{l+m+mi}{4}\PY{p}{]}\PY{p}{)}

\PY{k}{function}\PY{+w}{ }\PY{n}{kmeans\PYZus{}plot}\PY{p}{(}\PY{n}{X}\PY{p}{,}\PY{+w}{ }\PY{n}{K}\PY{p}{)}
\PY{+w}{    }\PY{n}{result}\PY{+w}{ }\PY{o}{=}\PY{+w}{ }\PY{n}{kmeans}\PY{p}{(}\PY{n}{X}\PY{o}{\PYZsq{}}\PY{p}{,}\PY{+w}{ }\PY{n}{K}\PY{p}{)}
\PY{+w}{    }\PY{n}{assignments}\PY{+w}{ }\PY{o}{=}\PY{+w}{ }\PY{n}{result}\PY{o}{.}\PY{n}{assignments}
\PY{+w}{    }\PY{n}{centroids}\PY{+w}{ }\PY{o}{=}\PY{+w}{ }\PY{n}{transpose}\PY{p}{(}\PY{n}{result}\PY{o}{.}\PY{n}{centers}\PY{p}{)}
\PY{+w}{    }\PY{n}{p}\PY{+w}{ }\PY{o}{=}\PY{+w}{ }\PY{n}{Plots}\PY{o}{.}\PY{n}{scatter}\PY{p}{(}\PY{n}{X}\PY{p}{[}\PY{o}{:}\PY{p}{,}\PY{+w}{ }\PY{l+m+mi}{1}\PY{p}{]}\PY{p}{,}\PY{+w}{ }\PY{n}{X}\PY{p}{[}\PY{o}{:}\PY{p}{,}\PY{+w}{ }\PY{l+m+mi}{2}\PY{p}{]}\PY{p}{,}\PY{+w}{ }\PY{n}{color}\PY{o}{=}\PY{n}{assignments}\PY{p}{,}\PY{+w}{ }\PY{n}{legend}\PY{o}{=}\PY{n+nb}{false}\PY{p}{,}\PY{+w}{ }\PY{n}{title}\PY{o}{=}\PY{l+s}{\PYZdq{}}\PY{l+s}{K\PYZhy{}means Clustering with K = }\PY{l+s+si}{\PYZdl{}K}\PY{l+s}{\PYZdq{}}\PY{p}{)}
\PY{+w}{    }\PY{n}{Plots}\PY{o}{.}\PY{n}{scatter!}\PY{p}{(}\PY{n}{centroids}\PY{p}{[}\PY{o}{:}\PY{p}{,}\PY{+w}{ }\PY{l+m+mi}{1}\PY{p}{]}\PY{p}{,}\PY{+w}{ }\PY{n}{centroids}\PY{p}{[}\PY{o}{:}\PY{p}{,}\PY{+w}{ }\PY{l+m+mi}{2}\PY{p}{]}\PY{p}{,}\PY{+w}{ }\PY{n}{marker}\PY{o}{=}\PY{p}{(}\PY{l+s+ss}{:x}\PY{p}{,}\PY{+w}{ }\PY{l+m+mi}{10}\PY{p}{)}\PY{p}{,}\PY{+w}{ }\PY{n}{linewidth}\PY{o}{=}\PY{l+m+mi}{3}\PY{p}{,}\PY{+w}{ }\PY{n}{color}\PY{o}{=}\PY{l+s+ss}{:black}\PY{p}{,}\PY{+w}{ }\PY{n}{label}\PY{o}{=}\PY{l+s}{\PYZdq{}}\PY{l+s}{Centroids}\PY{l+s}{\PYZdq{}}\PY{p}{)}
\PY{+w}{    }\PY{k}{return}\PY{+w}{ }\PY{n}{p}
\PY{k}{end}

\PY{n}{K}\PY{+w}{ }\PY{o}{=}\PY{+w}{ }\PY{p}{[}\PY{l+m+mi}{1}\PY{p}{,}\PY{+w}{ }\PY{l+m+mi}{3}\PY{p}{,}\PY{+w}{ }\PY{l+m+mi}{5}\PY{p}{,}\PY{+w}{ }\PY{l+m+mi}{10}\PY{p}{]}\PY{+w}{  }\PY{c}{\PYZsh{} modified range for K}

\PY{n}{p1}\PY{+w}{ }\PY{o}{=}\PY{+w}{ }\PY{n}{kmeans\PYZus{}plot}\PY{p}{(}\PY{n}{X}\PY{p}{,}\PY{+w}{ }\PY{n}{K}\PY{p}{[}\PY{l+m+mi}{1}\PY{p}{]}\PY{p}{)}
\PY{n}{p2}\PY{+w}{ }\PY{o}{=}\PY{+w}{ }\PY{n}{kmeans\PYZus{}plot}\PY{p}{(}\PY{n}{X}\PY{p}{,}\PY{+w}{ }\PY{n}{K}\PY{p}{[}\PY{l+m+mi}{2}\PY{p}{]}\PY{p}{)}
\PY{n}{p3}\PY{+w}{ }\PY{o}{=}\PY{+w}{ }\PY{n}{kmeans\PYZus{}plot}\PY{p}{(}\PY{n}{X}\PY{p}{,}\PY{+w}{ }\PY{n}{K}\PY{p}{[}\PY{l+m+mi}{3}\PY{p}{]}\PY{p}{)}
\PY{n}{p4}\PY{+w}{ }\PY{o}{=}\PY{+w}{ }\PY{n}{kmeans\PYZus{}plot}\PY{p}{(}\PY{n}{X}\PY{p}{,}\PY{+w}{ }\PY{n}{K}\PY{p}{[}\PY{l+m+mi}{4}\PY{p}{]}\PY{p}{)}
\PY{+w}{    }
\PY{n}{Plots}\PY{o}{.}\PY{n}{plot}\PY{p}{(}\PY{n}{p1}\PY{p}{,}\PY{+w}{ }\PY{n}{p2}\PY{p}{,}\PY{+w}{ }\PY{n}{p3}\PY{p}{,}\PY{+w}{ }\PY{n}{p4}\PY{p}{,}\PY{+w}{ }\PY{n}{layout}\PY{o}{=}\PY{p}{(}\PY{l+m+mi}{2}\PY{p}{,}\PY{+w}{ }\PY{l+m+mi}{2}\PY{p}{)}\PY{p}{,}\PY{+w}{ }\PY{n}{size}\PY{o}{=}\PY{p}{(}\PY{l+m+mi}{800}\PY{p}{,}\PY{+w}{ }\PY{l+m+mi}{800}\PY{p}{)}\PY{p}{)}
\end{Verbatim}
\end{tcolorbox}
 
            
\prompt{Out}{outcolor}{9}{}
    
    \begin{center}
    \adjustimage{max size={0.9\linewidth}{0.9\paperheight}}{Final Project_files/Final Project_50_0.pdf}
    \end{center}
    { \hspace*{\fill} \\}
    

    After running the code and visualizing the results, discuss the
importance of selecting an appropriate value for K in the K-means
clustering algorithm. Consider how varying K impacts the clustering
results and the potential implications for real-world applications.

    {Mini-Homework}

    In this mini-homework, you will solve two questions related to the
K-means clustering algorithm. The questions are designed to be at a
level similar to the questions you have been solving throughout the
course.

    \textbf{Question 1:} You are given a dataset with the following 2D
points:

    \begin{tcolorbox}[breakable, size=fbox, boxrule=1pt, pad at break*=1mm,colback=cellbackground, colframe=cellborder]
\prompt{In}{incolor}{10}{\boxspacing}
\begin{Verbatim}[commandchars=\\\{\}]
\PY{n}{A}\PY{+w}{ }\PY{o}{=}\PY{+w}{ }\PY{p}{[}\PY{l+m+mi}{1}\PY{p}{,}\PY{+w}{ }\PY{l+m+mi}{2}\PY{p}{]}
\PY{n}{B}\PY{+w}{ }\PY{o}{=}\PY{+w}{ }\PY{p}{[}\PY{l+m+mi}{2}\PY{p}{,}\PY{+w}{ }\PY{l+m+mi}{1}\PY{p}{]}
\PY{n}{C}\PY{+w}{ }\PY{o}{=}\PY{+w}{ }\PY{p}{[}\PY{l+m+mi}{6}\PY{p}{,}\PY{+w}{ }\PY{l+m+mi}{5}\PY{p}{]}
\PY{n}{D}\PY{+w}{ }\PY{o}{=}\PY{+w}{ }\PY{p}{[}\PY{l+m+mi}{5}\PY{p}{,}\PY{+w}{ }\PY{l+m+mi}{6}\PY{p}{]}
\end{Verbatim}
\end{tcolorbox}

            \begin{tcolorbox}[breakable, size=fbox, boxrule=.5pt, pad at break*=1mm, opacityfill=0]
\prompt{Out}{outcolor}{10}{\boxspacing}
\begin{Verbatim}[commandchars=\\\{\}]
2-element Vector\{Int64\}:
 5
 6
\end{Verbatim}
\end{tcolorbox}
        
    Using the K-means clustering algorithm with K = 2, manually perform one
iteration of the algorithm, assuming the initial centroids are
\(μ1 = A\) and \(μ2 = C\). Provide the updated centroids after one
iteration.

    \textbf{Solution:}

    \begin{enumerate}
\def\labelenumi{\arabic{enumi}.}
\tightlist
\item
  Assign each point to the nearest centroid:
\end{enumerate}

\begin{itemize}
\tightlist
\item
  \(A\) and \(B\) are closer to \(μ1\)
\item
  \(C\) and \(D\) are closer to \(μ2\)
\end{itemize}

\begin{enumerate}
\def\labelenumi{\arabic{enumi}.}
\setcounter{enumi}{1}
\tightlist
\item
  Update the centroids by computing the mean of the points assigned to
  each centroid:
\end{enumerate}

\begin{itemize}
\tightlist
\item
  \(μ1_{new} = (A + B) / 2 = [1.5, 1.5]\)
\item
  \(μ2_{new} = (C + D) / 2 = [5.5, 5.5]\)
\end{itemize}

    After one iteration, the updated centroids are \(μ1_{new} = [1.5, 1.5]\)
and \(μ2_{new} = [5.5, 5.5]\).

    \textbf{Question 2:} Consider the same dataset as in Question 1.
Implement the K-means clustering algorithm in Julia for \(K = 2\), using
the initial centroids \(μ1 = A\) and \(μ2 = C\). How many iterations
does it take for the algorithm to converge, and what are the final
centroids?

    \textbf{Solution:}

    \begin{tcolorbox}[breakable, size=fbox, boxrule=1pt, pad at break*=1mm,colback=cellbackground, colframe=cellborder]
\prompt{In}{incolor}{11}{\boxspacing}
\begin{Verbatim}[commandchars=\\\{\}]
\PY{c}{\PYZsh{} Define the dataset}
\PY{n}{X}\PY{+w}{ }\PY{o}{=}\PY{+w}{ }\PY{p}{[}\PY{l+m+mi}{1}\PY{+w}{ }\PY{l+m+mi}{2}\PY{p}{;}\PY{+w}{ }\PY{l+m+mi}{2}\PY{+w}{ }\PY{l+m+mi}{1}\PY{p}{;}\PY{+w}{ }\PY{l+m+mi}{6}\PY{+w}{ }\PY{l+m+mi}{5}\PY{p}{;}\PY{+w}{ }\PY{l+m+mi}{5}\PY{+w}{ }\PY{l+m+mi}{6}\PY{p}{]}

\PY{c}{\PYZsh{} Set the indices of initial centroids}
\PY{n}{init\PYZus{}centroid\PYZus{}indices}\PY{+w}{ }\PY{o}{=}\PY{+w}{ }\PY{p}{[}\PY{l+m+mi}{1}\PY{p}{,}\PY{+w}{ }\PY{l+m+mi}{3}\PY{p}{]}

\PY{c}{\PYZsh{} Perform K\PYZhy{}means clustering with K = 2 and initial centroids}
\PY{n}{K}\PY{+w}{ }\PY{o}{=}\PY{+w}{ }\PY{l+m+mi}{2}
\PY{n}{result}\PY{+w}{ }\PY{o}{=}\PY{+w}{ }\PY{n}{kmeans}\PY{p}{(}\PY{n}{X}\PY{o}{\PYZsq{}}\PY{p}{,}\PY{+w}{ }\PY{n}{K}\PY{p}{,}\PY{+w}{ }\PY{n}{init}\PY{o}{=}\PY{n}{init\PYZus{}centroid\PYZus{}indices}\PY{p}{)}

\PY{c}{\PYZsh{} Get the number of iterations and final centroids}
\PY{n}{iterations}\PY{+w}{ }\PY{o}{=}\PY{+w}{ }\PY{n}{result}\PY{o}{.}\PY{n}{iterations}
\PY{n}{final\PYZus{}centroids}\PY{+w}{ }\PY{o}{=}\PY{+w}{ }\PY{n}{transpose}\PY{p}{(}\PY{n}{result}\PY{o}{.}\PY{n}{centers}\PY{p}{)}

\PY{n}{println}\PY{p}{(}\PY{l+s}{\PYZdq{}}\PY{l+s}{Number of iterations: }\PY{l+s+si}{\PYZdl{}iterations}\PY{l+s}{\PYZdq{}}\PY{p}{)}
\PY{n}{println}\PY{p}{(}\PY{l+s}{\PYZdq{}}\PY{l+s}{Final centroids: }\PY{l+s+si}{\PYZdl{}final\PYZus{}centroids}\PY{l+s}{\PYZdq{}}\PY{p}{)}
\end{Verbatim}
\end{tcolorbox}

    \begin{Verbatim}[commandchars=\\\{\}]
Number of iterations: 2
Final centroids: [1.5 1.5; 5.5 5.5]
    \end{Verbatim}

    After running the code, you will find that the algorithm converges in 1
iteration, with final centroids \(μ1 = [1.5, 1.5]\) and
\(μ2 = [5.5, 5.5]\).


    % Add a bibliography block to the postdoc
    
    
    
\end{document}
